% Options for packages loaded elsewhere
\PassOptionsToPackage{unicode}{hyperref}
\PassOptionsToPackage{hyphens}{url}
%
\documentclass[
]{article}
\usepackage{lmodern}
\usepackage{amssymb,amsmath}
\usepackage{ifxetex,ifluatex}
\ifnum 0\ifxetex 1\fi\ifluatex 1\fi=0 % if pdftex
  \usepackage[T1]{fontenc}
  \usepackage[utf8]{inputenc}
  \usepackage{textcomp} % provide euro and other symbols
\else % if luatex or xetex
  \usepackage{unicode-math}
  \defaultfontfeatures{Scale=MatchLowercase}
  \defaultfontfeatures[\rmfamily]{Ligatures=TeX,Scale=1}
\fi
% Use upquote if available, for straight quotes in verbatim environments
\IfFileExists{upquote.sty}{\usepackage{upquote}}{}
\IfFileExists{microtype.sty}{% use microtype if available
  \usepackage[]{microtype}
  \UseMicrotypeSet[protrusion]{basicmath} % disable protrusion for tt fonts
}{}
\makeatletter
\@ifundefined{KOMAClassName}{% if non-KOMA class
  \IfFileExists{parskip.sty}{%
    \usepackage{parskip}
  }{% else
    \setlength{\parindent}{0pt}
    \setlength{\parskip}{6pt plus 2pt minus 1pt}}
}{% if KOMA class
  \KOMAoptions{parskip=half}}
\makeatother
\usepackage{xcolor}
\IfFileExists{xurl.sty}{\usepackage{xurl}}{} % add URL line breaks if available
\IfFileExists{bookmark.sty}{\usepackage{bookmark}}{\usepackage{hyperref}}
\hypersetup{
  pdftitle={Home},
  pdfauthor={N. Pollet},
  hidelinks,
  pdfcreator={LaTeX via pandoc}}
\urlstyle{same} % disable monospaced font for URLs
\usepackage[margin=1in]{geometry}
\usepackage{graphicx}
\makeatletter
\def\maxwidth{\ifdim\Gin@nat@width>\linewidth\linewidth\else\Gin@nat@width\fi}
\def\maxheight{\ifdim\Gin@nat@height>\textheight\textheight\else\Gin@nat@height\fi}
\makeatother
% Scale images if necessary, so that they will not overflow the page
% margins by default, and it is still possible to overwrite the defaults
% using explicit options in \includegraphics[width, height, ...]{}
\setkeys{Gin}{width=\maxwidth,height=\maxheight,keepaspectratio}
% Set default figure placement to htbp
\makeatletter
\def\fps@figure{htbp}
\makeatother
\setlength{\emergencystretch}{3em} % prevent overfull lines
\providecommand{\tightlist}{%
  \setlength{\itemsep}{0pt}\setlength{\parskip}{0pt}}
\setcounter{secnumdepth}{-\maxdimen} % remove section numbering
\ifluatex
  \usepackage{selnolig}  % disable illegal ligatures
\fi

\title{Home}
\author{N. Pollet}
\date{17/03/2024}

\begin{document}
\maketitle

\begin{center}\rule{0.5\linewidth}{0.5pt}\end{center}

\hypertarget{introduction}{%
\section{Introduction}\label{introduction}}

L'école thématique ``Genbarque Guyane'' du CEBA se focalisera sur
l'analyse de la biodiversité par des approches de métagénomique, plus
spécifiquement sur l'utilisation des approches de séquençage ONT
incluant l'analyse bioinformatique.

\hypertarget{expuxe9rimentations}{%
\section{Expérimentations}\label{expuxe9rimentations}}

Le volet expérimental au camp des Nouragues couvre les aspects de
génétique moléculaire allant de l'extraction de l'ADN ou ARN à la
construction de banques et séquençage sur flongle.

\hypertarget{bioinformatique}{%
\section{Bioinformatique}\label{bioinformatique}}

Le volet analyse bioinformatique couvre les parties de basecalling,
démultiplexage, contrôle qualité, assemblage de consensus pour le
séquençage d'amplicons, analyse de metabarcoding.

\hypertarget{organisation}{%
\section{Organisation}\label{organisation}}

Bonjour à tous et à toutes,

Notre école thématique arrive à grands pas !

Nous vous attendons tous ce mardi 02 avril entre 8h00 et 8h30 au Centre
de Recherche de Montabo dans la grande salle de réunion du bâtiment A.
Un petit déjeuner (viennoiseries, café, etc) vous attendra!

L'idéal serait de maximiser le covoiturage au sein des équipes afin de
limiter le nombre de voitures stationnées sur le centre de recherche de
Montabo. Vous pouvez garer les véhicules le long de la clôture du
sentier de Montabo.

\textbf{Venez avec votre déjeuner (sandwich, salade, eau etc) que vous
mangerez dans le bus, car nous serons dans les transports sur le midi.}

\textbf{Veuillez limiter le volume de vos bagages, car l'espace est
limité sur les hélicoptères. Veuillez prévoir un 1 sac type randonnée
\textasciitilde50-60L/touque/petite valise par personne + 1 item
personnel en cabine.}

L'équipe de la station des Nouragues vous fera parvenir sous peu des
informations clés pour la préparation de la mission.

Logistique des déplacements:

\hypertarget{mardi-02-avril}{%
\subsection{Mardi 02 Avril}\label{mardi-02-avril}}

\textbf{8:30 - 11:30} Accueil au Centre de Montabo dans la grande salle
de réunion.

\begin{verbatim}
  Tour de table des présentations, introduction de l’école
\end{verbatim}

\textbf{11:40 - 13:40} Départ du Centre de Recherche de Montabo
-\textgreater{} Bus -\textgreater{} Helipad de la piste de Belizon.

\textbf{14:15 - 16:00} Rotation de l'hélicoptère entre Belizon et le
Camp Inselbeg.

\textbf{16:00 - 17:00} Installation au camp Inselberg.

\textbf{17:00 - 18:00} Présentation de la station des Nouragues par
Élodie Schloesing et Cyril Gaertner.

\hypertarget{mercredi-03-avril-au-vendredi-05-avril}{%
\subsection{Mercredi 03 avril au Vendredi 05
avril}\label{mercredi-03-avril-au-vendredi-05-avril}}

\begin{verbatim}
  Déroulement de la phase pratique de l’école thématique.
\end{verbatim}

\hypertarget{samedi-06-avril}{%
\subsection{Samedi 06 avril}\label{samedi-06-avril}}

\textbf{8:00 - 12:00} Module de l'école thématique.

\textbf{13:00 -\textgreater{} 14:30} Rangement du camp.

\textbf{15:00 -\textgreater{} 17:20} Rotation de l'hélicoptère entre le
Camp Inselbeg et Belizon.

\textbf{17:30 -\textgreater{} 19:00} Belizon -\textgreater{} Bus
-\textgreater{} Centre de Recherche de Montabo.

\textbf{Échantillons à séquencer}

Pour ceux qui ont des échantillons qu'ils voudraient analyser durant
l'école thématique, une glacière sera mise à disposition afin de
préserver les échantillons durant le transport. \textbf{Seuls des
échantillons de niveau L1 sont éligibles}.

Veuillez-vous assurer que vos échantillons remplissent les critères
suivants :

\begin{verbatim}
   ADN pour faire des PCR CO1 etc: dilution à 10 ng/µL et QC avec Qubit/Nanodrop ou gel

   PCR pour séquençage: 200 ng de produit de PCR purifié à 10-20 ng/µL minimum et QC avec Qubit/Nanodrop ou gel. De plus, nous aurons besoin de la taille des amplicons et de la séquence des oligonucléotides utilisés pour l'atelier bioinformatique.

   cDNA : 100 ng de produit de cDNA purifié à 50 ng/µL et QC avec Qubit/Nanodrop ou gel
\end{verbatim}

\hypertarget{lundi-08-avril-au-mercredi-10-avril-de-8h30-uxe0-17h00}{%
\subsection{Lundi 08 avril au mercredi 10 avril de 8h30 à
17h00}\label{lundi-08-avril-au-mercredi-10-avril-de-8h30-uxe0-17h00}}

\begin{verbatim}
  Déroulement du Module bioinformatique dans la grande salle du Centre de recherche de Montabo
\end{verbatim}

Il faudra installer les logiciels suivants :

-\href{https://mobaxterm.mobatek.net/}{MobaXterm} pour windows.

-\href{https://filezilla-project.org/download.php?type=client}{FileZilla}

\hypertarget{la-fine-uxe9quipe}{%
\subsection{La fine équipe}\label{la-fine-uxe9quipe}}

Mathieu Chouteau, LEEISA

Sourakhata Tirera, Institut Pasteur de la Guyane

Nicolas Pollet, EGCE

Isabelle Clavereau, EGCE

\end{document}
